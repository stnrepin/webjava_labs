\documentclass[a4paper,14pt]{extarticle}
\usepackage{rotating}
\usepackage{verbatim}
%%% Преамбула %%%

\usepackage{fontspec}
\usepackage{xunicode}
\usepackage{xltxtra}
\usepackage{pdfpages}

%%\usepackage[europeanresistors,americaninductors,
%%            americancurrents,americanvoltage]{circuitikz}



%
% Шрифты
%
\defaultfontfeatures{Ligatures=TeX}
\newfontfamily{\cyrillicfont}{Tinos}
\newfontfamily{\cyrillicfontrm}{Tinos}
\newfontfamily{\cyrillicfonttt}{Cousine}
\newfontfamily{\cyrillicfontsf}{Arimo}
\setmonofont{Cousine}[Scale=MatchLowercase]

% Красная строка после заголовка
%%\usepackage{indentfirst}

% Русский язык
\usepackage{polyglossia}
\setdefaultlanguage{russian}

\usepackage{amssymb,amsfonts,amsmath} % Математика
\numberwithin{equation}{section} % Формула вида секция.номер

\renewcommand{\footnotesize}{\small}


% Оформление библиографии и подрисуночных записей через точку
\makeatletter
\renewcommand*{\@biblabel}[1]{\hfill#1.}
\renewcommand*\l@section{\@dottedtocline{1}{1em}{1em}}
\renewcommand{\thefigure}{\thesection.\arabic{figure}} % Формат рисунка секция.номер
\renewcommand{\thetable}{\thesection.\arabic{table}} % Формат таблицы секция.номер
\def\redeflsection{\def\l@section{\@dottedtocline{1}{0em}{10em}}}
\makeatother

% Полуторный межстрочный интервал
%%\renewcommand{\baselinestretch}{1.4}
\setlength{\parskip}{3ex}
\setlength{\parindent}{0cm} % Абзацный отступ

\sloppy
\hyphenpenalty=1000
\clubpenalty=10000
\widowpenalty=10000

% Отступы у страниц
\usepackage{geometry}
\geometry{left=3cm}
\geometry{top=2cm}
\geometry{right=1.5cm}
\geometry{bottom=2cm}

%
% Списки
%
%\usepackage{enumerate} Если вдруг сломается что-то
\usepackage{enumitem}
% TODO: Проверить topsep vs nolistsep
\setlist[enumerate,itemize]{nolistsep,topsep=0pt,parsep=0.5cm}

%%\makeatletter
%%\AddEnumerateCounter{\asbuk}{\russian@alph}{щ}
%%\makeatother
%%\setlist{nolistsep} % Нет отступов между пунктами списка
\renewcommand{\labelitemi}{--} % Маркет списка --
%%\renewcommand{\labelenumi}{\asbuk{enumi})} % Список второго уровня
%%\renewcommand{\labelenumii}{\arabic{enumii})} % Список третьего уровня



%
% Содержание
%
\usepackage{tocloft}
\usepackage[dotinlabels]{titletoc}
\renewcommand{\cfttoctitlefont}
    {\normalfont\bfseries\hspace{0.38\textwidth}\MakeTextUppercase}
\renewcommand{\cftsecfont}{\hspace{0pt}}
% Точки для секций в содержании
%%\renewcommand\cftsecleader{\cftdotfill{\cftdotsep}}
\renewcommand\cftsecpagefont{\mdseries}
\setcounter{tocdepth}{3}



%
% Рисунки
%

\usepackage{graphicx}

% Формат подрисуночных записей
\usepackage{chngcntr}
\counterwithin{figure}{section}

% Формат подрисуночных надписей
\RequirePackage{caption}
\DeclareCaptionLabelSeparator{defffis}{ }
\captionsetup[figure]{justification=centering, labelsep=defffis, format=plain}
\captionsetup[table]{justification=raggedright, labelsep=defffis, format=plain, singlelinecheck=false}
\addto\captionsrussian{\renewcommand{\figurename}{Рис.}}

% Пользовательские функции
\newcommand{\addimg}[4]{ % Добавление одного рисунка
    \begin{figure}
        \centering
        \includegraphics[width=#2\linewidth]{#1}
        \caption{#3} \label{#4}
    \end{figure}
}
\newcommand{\addimghere}[4]{ % Добавить рисунок непосредственно в это место
    \begin{figure}[H]
        \centering
        \includegraphics[width=#2\linewidth]{#1}
        \caption{#3} \label{#4}
    \end{figure}
}
\newcommand{\addanonimghere}[2]{
    \begin{figure}[H]
        \centering
        \includegraphics[width=#2\linewidth]{#1}
    \end{figure}
}
\newcommand{\addimgherev}[5]{ % Добавить рисунок непосредственно в это место
    \begin{figure}[H]
        \centering
        \includegraphics[width=#3\linewidth]{#1}
        \includegraphics[width=#3\linewidth]{#2}
        \caption{#4} \label{#5}
    \end{figure}
}
\newcommand{\addtwoimghere}[5]{ % Вставка двух рисунков
    \begin{figure}[H]
        \centering
        \includegraphics[width=#3\linewidth]{#1}
        \hfill
        \includegraphics[width=#3\linewidth]{#2}
        \caption{#4} \label{#5}
    \end{figure}
}
\newcommand{\addimgapp}[2]{ % Это костыль для приложения Б
    \begin{figure}[H]
        \centering
        \includegraphics[width=1\linewidth]{#1}
        \caption*{#2}
    \end{figure}
}
\newcommand{\addcircform}[5]{
    \hspace*{-1cm}%
    \begin{minipage}{#4\textwidth}
        \addimghere{#1}{}{#2}{3}
        \vspace{0.3cm}
    \end{minipage}%
    \hfill%
    \begin{minipage}{#5\textwidth}
        #3
        \vspace{3.5cm}%
    \end{minipage}
}



%
% Таблицы
%
\usepackage{float}
\usepackage{tabularx}
\usepackage{longtable}
\usepackage{multirow}
\usepackage{tabu}


% Хаки для таблиц
\newcommand{\centmultirow}[3]{%
    \multicolumn{1}{|c|}{\multirow{#1}{#2}{\centering #3}}%
}
\newcolumntype{P}[1]{>{\centering\arraybackslash}p{#1}}
\newcolumntype{C}{>{\centering\arraybackslash}X}

\newcommand{\inputtabhere}[3]{%
    \begin{table}[h]
        \renewcommand{\arraystretch}{1.30}
        \renewcommand{\tabularxcolumn}[1]{m{##1}}
        \input{#1}
        \caption{#2}
        \label{#3}
    \end{table}
}



%
% Секции
%

% Заголовки секций в оглавлении в верхнем регистре
\usepackage{textcase}
\makeatletter
\let\oldcontentsline\contentsline
\def\contentsline#1#2{
    \expandafter\ifx\csname l@#1\endcsname\l@section
        \expandafter\@firstoftwo
    \else
        \expandafter\@secondoftwo
    \fi
    {\oldcontentsline{#1}{#2}}
    {\oldcontentsline{#1}{#2}}
}
\makeatother

% Оформление заголовков
\usepackage[compact,explicit]{titlesec}
\titleformat{\section}{}{}{0pt}
    {\centering\textbf{\thesection.\space\MakeTextUppercase{#1}}}
\titleformat{\subsection}[block]{\vspace{1em}}{}{0pt}
    {\centering\textbf{\thesubsection.\space#1}}
\titleformat{\subsubsection}[block]{\vspace{1em}\normalsize}{}{0pt}
    {\centering\textbf{\thesubsubsection.\space#1\vspace{1em}}}
\titleformat{\paragraph}[block] {\normalsize}{}{12.5mm}
    {\MakeTextUppercase{#1}}

% Секции без номеров (введение, заключение...), вместо section*{}
\newcommand{\anonsection}[1]{
    \vspace{1.5em}
    \phantomsection % Корректный переход по ссылкам в содержании
    \paragraph{\normalfont\bfseries\centerline{{#1}}}
    \addcontentsline{toc}{section}{#1}
}

% Секции для приложений
\newcommand{\appsection}[1]{
    \phantomsection
    \paragraph{\centerline{{#1}}}
    \addcontentsline{toc}{section}{\uppercase{#1}}
}



%
% Остальное
%

%\usepackage{listings} % Оформление исходного кода
%\lstset{
%    language=C++,
%    title=\lstname,
%    basicstyle=\small\ttfamily, % Размер и тип шрифта
%    breaklines=\true, % Перенос строк
%    tabsize=2, % Размер табуляции
%    literate={--}{{-{}-}}2, % Корректно отображать двойной дефис
%    texcl=\true,
%    showspaces=\false,
%    showstringspaces=\false,
%    extendedchars=\true,
%}

% Гиперссылки
%%
%% ВНИМАНИЕ!
%% \usepackage{hyperref} должен идти после всех остальных \usepackage.
%%
\usepackage{ulem}
\usepackage{hyperref}
\urlstyle{same}
\hypersetup{
    colorlinks, urlcolor=blue,
    linkcolor={black}, citecolor={black}, filecolor={black},
    pdfauthor={Репин Степан}, pdfborderstyle={/S/U/W 1}, allbordercolors=0 0 1,
}


% Простой текст в формулах
\newcommand{\tc}[1]{\text{#1}}
% Единицы измерения
\newcommand{\un}[1]{\tc{\ \ #1}}

% Различные физические обозначения.
\newcommand{\Lap}[0]{\mathcal{L}}
\newcommand{\ILap}[0]{\Lap^{-1}}
\newcommand{\Hev}[0]{\delta_1}

% Настройка CircuiTikz
%%\ctikzset{voltage/distance from node=1cm}
%%\ctikzset{bipoles/thickness=1}



\usepackage[outputdir=out,cache=false]{minted}

\begin{document}

\begin{titlepage}
\newgeometry{left=1.18in,top=0.60in,right=0.39in,bottom=0.59in}
\begin{center}
    \uppercase{\textbf{Минобрнауки России\\
            Санкт-Петербургский государственный\\
            электротехнический университет\\
            «ЛЭТИ» им. В.И.Ульянова (Ленина)
    }}
    \vspace{0.25cm}

    \textbf{Кафедра ВТ}
    \vfill

    \uppercase{\textbf{\large{
        Отчет
    }}}
    \\
    \textbf{\large{
      по лабораторной работе №4\\
      по дисциплине «Java программирование интернет-приложений»\\
      Тема: Интернационализация WEB-приложений
      \vspace{0.5cm}
    }}
  \bigskip
\end{center}
\vfill
\vfill

\noindent

\begin{tabularx}{\textwidth}{@{}lcXr}
    \phantom{Студенты гр. 8307} & \hspace{1.6cm} & \rule{5cm}{1pt} & Репин С.А.
\end{tabularx}

\vspace{0.5cm}

\noindent
\begin{tabularx}{\textwidth}{@{}lcXr}
    Преподаватель & \hspace{2.3cm} & \rule{5cm}{1pt} & Павловский М.Г.
\end{tabularx}

\hfill \break
\hfill \break

\begin{center}
  Санкт-Петербург\\2022
\end{center}

\end{titlepage}



\renewcommand*{\thepage}{}
\tableofcontents
\clearpage
\renewcommand*{\thepage}{\arabic{page}}

\setcounter{page}{3}

\section{Цель работы}

Знакомство со способами отображения данных на различных языках при
использовании файлов ресурсов.

\section{Выполнение работы}

В приложение, разработанное в предыдущей работе, были добавлены ресурсы,
необхимые для интернационализации приложения на русский и английский языки.

\addimghere{res/1.png}{1}{Указание неверного языка}{}
\addimghere{res/2.png}{1}{Приложение на английском языке}{}
\addimghere{res/3.png}{1}{Приложение на русском языке}{}

\section{Вывод}

В ходе выполнения лабораторной работы были получены практические навыки
работы с ресурсами приложений и выполнения на их основе интернационализации
веб-приложений.

\anonsection{Исходный код сервлета}

\begin{minted}[fontsize=\footnotesize]{java}
package app.servlet;

import jakarta.servlet.*;
import jakarta.servlet.http.*;
import jakarta.servlet.annotation.*;

import java.io.IOException;
import java.io.PrintWriter;
import java.util.ArrayList;
import java.util.List;
import java.util.Locale;
import java.util.ResourceBundle;

/**
 * Лекарство в аптеке.
 */
class Medicine {
    public int id;
    public String name;
    public int quantity;
    public int unitPrice;

    /**
     * Конкструктор.
     * @param id ID
     * @param name Имя
     * @param quantity Количество на складе
     * @param unitPrice Цена единицы
     */
    public Medicine(int id, String name, int quantity, int unitPrice) {
        this.id = id;
        this.name = name;
        this.quantity = quantity;
        this.unitPrice = unitPrice;
    }

    /**
     * Возвращает список строковых значений полей класса.
     * @return Список строк
     */
    public List<String> fieldsToStringList() {
        return List.of(
            String.valueOf(this.id),
            this.name,
            String.valueOf(this.quantity),
            String.valueOf(this.unitPrice)
        );
    }
}

/**
 * Сервлет, отображающий список лекарсв.
 */
@WebServlet(name = "MedicineList", value = "/MedicineList")
public class MedicineList extends HttpServlet {
    /**
     * Список лекарств.
     */
    private final List<Medicine> medicines;

    /**
     * Конкструктор.
     */
    public MedicineList() {
        super();
        medicines = new ArrayList<>();
        medicines.add(new Medicine(0, "AAA", 10, 100));
        medicines.add(new Medicine(1, "BBB", 32, 200));
        medicines.add(new Medicine(2, "CCC", 1, 220));
        medicines.add(new Medicine(3, "DDD", 56, 1340));
        medicines.add(new Medicine(4, "EEE", 23, 50));
    }

    /**
     * Выполняет GET и POST HTTP-запросы.
     *
     * @param request Запрос к сервлету
     * @param response Ответ сервлета
     * @throws ServletException Внутренняя ошибка
     * @throws IOException Ошибка ввода вывода
     */
    protected void processRequest(HttpServletRequest request,
                                  HttpServletResponse response)
            throws ServletException, IOException {

        var lang = request.getParameter("lang");
        if (lang == null) {
            response.sendError(HttpServletResponse.SC_NOT_ACCEPTABLE,
                    "Ожидался параметр lang");
            return;
        }
        if (!"en".equalsIgnoreCase(lang) && !"ru".equalsIgnoreCase(lang)) {
            response.sendError(HttpServletResponse.SC_NOT_ACCEPTABLE,
                    "Параметр lang может принимать значения en или ru");
            return;
        }

        var res = ResourceBundle.getBundle(
                                    "/MedicineList",
                                    new Locale(lang));
        request.setCharacterEncoding("utf-8");
        response.setContentType("text/html;charset=UTF-8");

        var max_price_str = request.getParameter("max-price");
        var max_price = (max_price_str == null)
                                ? 0
                                : Integer.parseInt(max_price_str);
        var table_content = buildTableHtml(max_price);

        var head_title = res.getString("head.title");
        var body_title = res.getString("body.title");
        var price_units = res.getString("price_units");
        var id = res.getString("table.id");
        var name = res.getString("table.name");
        var quantity = res.getString("table.quantity");
        var unit_price = res.getString("table.unit_price");

        try (PrintWriter out = response.getWriter()) {
            out.println(
                "<html>"
                + "<head><title>" + head_title + "</title></head>"
                + "<body>"
                        + "<h1>"
                            + body_title + " "
                            + max_price + " " + price_units
                        + "</h1>"
                        + "<table border='1'>"
                        + "<tr>"
                            + "<td><b>" + id + "</b></td>"
                            + "<td><b>" + name + "</b></td>"
                            + "<td><b>" + quantity + "</b></td>"
                            + "<td><b>" + unit_price + "</b></td>"
                        + "</tr>"
                        + table_content
                        + "</table>"
                + "</body" +
                "</html>"
            );
        }
    }

    /**
     * Создает содержимое таблицы для всех лекарств, стоимость единицы
     * которых меньше, чем `max_price`.
     * @param max_price Максимальная цена (не включая)
     * @return HTML-содержимое таблицы
     */
    private String buildTableHtml(int max_price) {
        StringBuilder sb = new StringBuilder();
        for (var med : this.medicines) {
            if (med.unitPrice >= max_price) {
                continue;
            }
            sb.append("<tr>");
            for (String field : med.fieldsToStringList()) {
                sb.append("<td>");
                sb.append(field);
                sb.append("</td>");
            }
            sb.append("</tr>");
        }
        return sb.toString();
    }

    /**
     * Выполняет GET HTTP-запрос.
     *
     * @param request Запрос к сервлету
     * @param response Ответ сервлета
     */
    @Override
    protected void doGet(HttpServletRequest request, HttpServletResponse response)
            throws ServletException, IOException {
        processRequest(request, response);
    }

    /**
     * Выполняет POST HTTP-запрос.
     *
     * @param request Запрос к сервлету
     * @param response Ответ сервлета
     */
    @Override
    protected void doPost(HttpServletRequest request, HttpServletResponse response)
            throws ServletException, IOException {
        processRequest(request, response);
    }
}
\end{minted}

\anonsection{Исходный код ресурсов (ангилийский язык)}

\begin{minted}[fontsize=\footnotesize]{java}
head.title = Medicine List
body.title = List of medicines cheaper than
price_units = rub/unit
table.id = ID
table.name = Name
table.quantity = Quantity
table.unit_price = Unit Price
\end{minted}

\anonsection{Исходный код ресурсов (русский язык)}

\begin{minted}[fontsize=\footnotesize]{java}
head.title = Список лекарств
body.title = Список лекарств дешевле
price_units = руб/шт
table.id = Идентификатор
table.name = Имя
table.quantity = Количество
table.unit_price = Цена за ед
\end{minted}

\end{document}

