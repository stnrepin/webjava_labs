\documentclass[a4paper,14pt]{extarticle}
\usepackage{rotating}
\usepackage{verbatim}
%%% Преамбула %%%

\usepackage{fontspec}
\usepackage{xunicode}
\usepackage{xltxtra}
\usepackage{pdfpages}

%%\usepackage[europeanresistors,americaninductors,
%%            americancurrents,americanvoltage]{circuitikz}



%
% Шрифты
%
\defaultfontfeatures{Ligatures=TeX}
\newfontfamily{\cyrillicfont}{Tinos}
\newfontfamily{\cyrillicfontrm}{Tinos}
\newfontfamily{\cyrillicfonttt}{Cousine}
\newfontfamily{\cyrillicfontsf}{Arimo}
\setmonofont{Cousine}[Scale=MatchLowercase]

% Красная строка после заголовка
%%\usepackage{indentfirst}

% Русский язык
\usepackage{polyglossia}
\setdefaultlanguage{russian}

\usepackage{amssymb,amsfonts,amsmath} % Математика
\numberwithin{equation}{section} % Формула вида секция.номер

\renewcommand{\footnotesize}{\small}


% Оформление библиографии и подрисуночных записей через точку
\makeatletter
\renewcommand*{\@biblabel}[1]{\hfill#1.}
\renewcommand*\l@section{\@dottedtocline{1}{1em}{1em}}
\renewcommand{\thefigure}{\thesection.\arabic{figure}} % Формат рисунка секция.номер
\renewcommand{\thetable}{\thesection.\arabic{table}} % Формат таблицы секция.номер
\def\redeflsection{\def\l@section{\@dottedtocline{1}{0em}{10em}}}
\makeatother

% Полуторный межстрочный интервал
%%\renewcommand{\baselinestretch}{1.4}
\setlength{\parskip}{3ex}
\setlength{\parindent}{0cm} % Абзацный отступ

\sloppy
\hyphenpenalty=1000
\clubpenalty=10000
\widowpenalty=10000

% Отступы у страниц
\usepackage{geometry}
\geometry{left=3cm}
\geometry{top=2cm}
\geometry{right=1.5cm}
\geometry{bottom=2cm}

%
% Списки
%
%\usepackage{enumerate} Если вдруг сломается что-то
\usepackage{enumitem}
% TODO: Проверить topsep vs nolistsep
\setlist[enumerate,itemize]{nolistsep,topsep=0pt,parsep=0.5cm}

%%\makeatletter
%%\AddEnumerateCounter{\asbuk}{\russian@alph}{щ}
%%\makeatother
%%\setlist{nolistsep} % Нет отступов между пунктами списка
\renewcommand{\labelitemi}{--} % Маркет списка --
%%\renewcommand{\labelenumi}{\asbuk{enumi})} % Список второго уровня
%%\renewcommand{\labelenumii}{\arabic{enumii})} % Список третьего уровня



%
% Содержание
%
\usepackage{tocloft}
\usepackage[dotinlabels]{titletoc}
\renewcommand{\cfttoctitlefont}
    {\normalfont\bfseries\hspace{0.38\textwidth}\MakeTextUppercase}
\renewcommand{\cftsecfont}{\hspace{0pt}}
% Точки для секций в содержании
%%\renewcommand\cftsecleader{\cftdotfill{\cftdotsep}}
\renewcommand\cftsecpagefont{\mdseries}
\setcounter{tocdepth}{3}



%
% Рисунки
%

\usepackage{graphicx}

% Формат подрисуночных записей
\usepackage{chngcntr}
\counterwithin{figure}{section}

% Формат подрисуночных надписей
\RequirePackage{caption}
\DeclareCaptionLabelSeparator{defffis}{ }
\captionsetup[figure]{justification=centering, labelsep=defffis, format=plain}
\captionsetup[table]{justification=raggedright, labelsep=defffis, format=plain, singlelinecheck=false}
\addto\captionsrussian{\renewcommand{\figurename}{Рис.}}

% Пользовательские функции
\newcommand{\addimg}[4]{ % Добавление одного рисунка
    \begin{figure}
        \centering
        \includegraphics[width=#2\linewidth]{#1}
        \caption{#3} \label{#4}
    \end{figure}
}
\newcommand{\addimghere}[4]{ % Добавить рисунок непосредственно в это место
    \begin{figure}[H]
        \centering
        \includegraphics[width=#2\linewidth]{#1}
        \caption{#3} \label{#4}
    \end{figure}
}
\newcommand{\addanonimghere}[2]{
    \begin{figure}[H]
        \centering
        \includegraphics[width=#2\linewidth]{#1}
    \end{figure}
}
\newcommand{\addimgherev}[5]{ % Добавить рисунок непосредственно в это место
    \begin{figure}[H]
        \centering
        \includegraphics[width=#3\linewidth]{#1}
        \includegraphics[width=#3\linewidth]{#2}
        \caption{#4} \label{#5}
    \end{figure}
}
\newcommand{\addtwoimghere}[5]{ % Вставка двух рисунков
    \begin{figure}[H]
        \centering
        \includegraphics[width=#3\linewidth]{#1}
        \hfill
        \includegraphics[width=#3\linewidth]{#2}
        \caption{#4} \label{#5}
    \end{figure}
}
\newcommand{\addimgapp}[2]{ % Это костыль для приложения Б
    \begin{figure}[H]
        \centering
        \includegraphics[width=1\linewidth]{#1}
        \caption*{#2}
    \end{figure}
}
\newcommand{\addcircform}[5]{
    \hspace*{-1cm}%
    \begin{minipage}{#4\textwidth}
        \addimghere{#1}{}{#2}{3}
        \vspace{0.3cm}
    \end{minipage}%
    \hfill%
    \begin{minipage}{#5\textwidth}
        #3
        \vspace{3.5cm}%
    \end{minipage}
}



%
% Таблицы
%
\usepackage{float}
\usepackage{tabularx}
\usepackage{longtable}
\usepackage{multirow}
\usepackage{tabu}


% Хаки для таблиц
\newcommand{\centmultirow}[3]{%
    \multicolumn{1}{|c|}{\multirow{#1}{#2}{\centering #3}}%
}
\newcolumntype{P}[1]{>{\centering\arraybackslash}p{#1}}
\newcolumntype{C}{>{\centering\arraybackslash}X}

\newcommand{\inputtabhere}[3]{%
    \begin{table}[h]
        \renewcommand{\arraystretch}{1.30}
        \renewcommand{\tabularxcolumn}[1]{m{##1}}
        \input{#1}
        \caption{#2}
        \label{#3}
    \end{table}
}



%
% Секции
%

% Заголовки секций в оглавлении в верхнем регистре
\usepackage{textcase}
\makeatletter
\let\oldcontentsline\contentsline
\def\contentsline#1#2{
    \expandafter\ifx\csname l@#1\endcsname\l@section
        \expandafter\@firstoftwo
    \else
        \expandafter\@secondoftwo
    \fi
    {\oldcontentsline{#1}{#2}}
    {\oldcontentsline{#1}{#2}}
}
\makeatother

% Оформление заголовков
\usepackage[compact,explicit]{titlesec}
\titleformat{\section}{}{}{0pt}
    {\centering\textbf{\thesection.\space\MakeTextUppercase{#1}}}
\titleformat{\subsection}[block]{\vspace{1em}}{}{0pt}
    {\centering\textbf{\thesubsection.\space#1}}
\titleformat{\subsubsection}[block]{\vspace{1em}\normalsize}{}{0pt}
    {\centering\textbf{\thesubsubsection.\space#1\vspace{1em}}}
\titleformat{\paragraph}[block] {\normalsize}{}{12.5mm}
    {\MakeTextUppercase{#1}}

% Секции без номеров (введение, заключение...), вместо section*{}
\newcommand{\anonsection}[1]{
    \vspace{1.5em}
    \phantomsection % Корректный переход по ссылкам в содержании
    \paragraph{\normalfont\bfseries\centerline{{#1}}}
    \addcontentsline{toc}{section}{#1}
}

% Секции для приложений
\newcommand{\appsection}[1]{
    \phantomsection
    \paragraph{\centerline{{#1}}}
    \addcontentsline{toc}{section}{\uppercase{#1}}
}



%
% Остальное
%

%\usepackage{listings} % Оформление исходного кода
%\lstset{
%    language=C++,
%    title=\lstname,
%    basicstyle=\small\ttfamily, % Размер и тип шрифта
%    breaklines=\true, % Перенос строк
%    tabsize=2, % Размер табуляции
%    literate={--}{{-{}-}}2, % Корректно отображать двойной дефис
%    texcl=\true,
%    showspaces=\false,
%    showstringspaces=\false,
%    extendedchars=\true,
%}

% Гиперссылки
%%
%% ВНИМАНИЕ!
%% \usepackage{hyperref} должен идти после всех остальных \usepackage.
%%
\usepackage{ulem}
\usepackage{hyperref}
\urlstyle{same}
\hypersetup{
    colorlinks, urlcolor=blue,
    linkcolor={black}, citecolor={black}, filecolor={black},
    pdfauthor={Репин Степан}, pdfborderstyle={/S/U/W 1}, allbordercolors=0 0 1,
}


% Простой текст в формулах
\newcommand{\tc}[1]{\text{#1}}
% Единицы измерения
\newcommand{\un}[1]{\tc{\ \ #1}}

% Различные физические обозначения.
\newcommand{\Lap}[0]{\mathcal{L}}
\newcommand{\ILap}[0]{\Lap^{-1}}
\newcommand{\Hev}[0]{\delta_1}

% Настройка CircuiTikz
%%\ctikzset{voltage/distance from node=1cm}
%%\ctikzset{bipoles/thickness=1}



\usepackage[outputdir=out,cache=false]{minted}

\begin{document}

\begin{titlepage}
\newgeometry{left=1.18in,top=0.60in,right=0.39in,bottom=0.59in}
\begin{center}
    \uppercase{\textbf{Минобрнауки России\\
            Санкт-Петербургский государственный\\
            электротехнический университет\\
            «ЛЭТИ» им. В.И.Ульянова (Ленина)
    }}
    \vspace{0.25cm}

    \textbf{Кафедра ВТ}
    \vfill

    \uppercase{\textbf{\large{
        Отчет
    }}}
    \\
    \textbf{\large{
      по лабораторной работе №4\\
      по дисциплине «Java программирование интернет-приложений»\\
      Тема: Интернационализация WEB-приложений
      \vspace{0.5cm}
    }}
  \bigskip
\end{center}
\vfill
\vfill

\noindent

\begin{tabularx}{\textwidth}{@{}lcXr}
    \phantom{Студенты гр. 8307} & \hspace{1.6cm} & \rule{5cm}{1pt} & Репин С.А.
\end{tabularx}

\vspace{0.5cm}

\noindent
\begin{tabularx}{\textwidth}{@{}lcXr}
    Преподаватель & \hspace{2.3cm} & \rule{5cm}{1pt} & Павловский М.Г.
\end{tabularx}

\hfill \break
\hfill \break

\begin{center}
  Санкт-Петербург\\2022
\end{center}

\end{titlepage}



\renewcommand*{\thepage}{}
\tableofcontents
\clearpage
\renewcommand*{\thepage}{\arabic{page}}

\setcounter{page}{3}

\section{Цель работы}

Знакомство с методами передачи информации между соединениями, открываемыми в
рамках одного сеанса работы пользователя

\section{Выполнение работы}

В проект добавлена страницы Login.jsp, которая содержит форму для ввода имени
пользователя и цвета. Эта страница передает GET-запросом данные в сервлет
LoginProcessor, который создает нужные cookie и переменные HTTP-сессии (имя
пользователя, цвет пользователя, число обращений к странице и дату последнего
обращения). Затем сервлет переадресует запрос на страницу Welcome.jsp, которая
читает созданные переменные и отображает соответствующий текст пользователю.

\addimghere{res/1_1.png}{1}{Ввод данных на страницу Login.jsp}{}
\addimghere{res/1_2.png}{1}{Результат на странице Welcome.jsp}{}
\addimghere{res/1_3.png}{1}{Получившиеся в итоге cookie}{}

\addimghere{res/2_1.png}{1}{Ввод данных на страницу Login.jsp}{}
\addimghere{res/2_2.png}{1}{Результат на странице Welcome.jsp}{}
\addimghere{res/2_3.png}{1}{Получившиеся в итоге cookie}{}

\addimghere{res/3_1.png}{1}{Ввод данных на страницу Login.jsp}{}
\addimghere{res/3_2.png}{1}{Результат на странице Welcome.jsp}{}
\addimghere{res/3_3.png}{1}{Получившиеся в итоге cookie}{}

\section{Вывод}

В ходе выполнения лабораторной работы были получены практические навыки
использования cookie и переменных HTTP-сессии в web-приложения на языке Java.
\anonsection{Исходный код Login.jsp}

\begin{minted}[fontsize=\footnotesize]{jsp}
<%@ page contentType="text/html;charset=UTF-8" language="java" %>
<html>
<head>
    <title>Login</title>
</head>
<body>
    <form action="LoginProcessor">
        <label for="username">Username:</label>
        <br>
        <input id="username" name="userName" autocomplete="off" />
        <br>
        <label for="color">Your color: </label>
        <br>
        <input id="color" name="userColor" autocomplete="off" />
        <br><br>
        <input type="submit" value="Login" />
    </form>
</body>
</html>
\end{minted}

\anonsection{Исходный код LoginProcessor.java}

\begin{minted}[fontsize=\footnotesize]{java}
package app.servlet;

import jakarta.servlet.http.*;
import jakarta.servlet.annotation.*;

import java.io.IOException;
import java.net.URLEncoder;
import java.nio.charset.StandardCharsets;
import java.time.Instant;
import java.util.Optional;

/**
 * Сервлет, обрабатывающий попытку входа пользователя.
 */
@WebServlet(name = "LoginProcessor", value = "/LoginProcessor")
public class LoginProcessor extends HttpServlet {
    /**
     * Обрабатывает GET-запрос.
     * @param request Объект запроса
     * @param response Объект ответа
     * @throws IOException Ошибка ввода/вывода
     */
    @Override
    protected void doGet(HttpServletRequest request,
        HttpServletResponse response)
            throws IOException {
        // Получаем HTTP-сессию.
        var session = request.getSession(true);

        response.setContentType("text/html; charset=UTF-8");
        request.setCharacterEncoding("utf-8");

        // Читаем параметры, переданные из формы.
        var userName = readParameter(request, response, "userName");
        var userColor = readParameter(request, response, "userColor");

        if (userName == null || userColor == null) {
            return;
        }

        // Создаем куки с именем пользователя и его цветом.
        createCookie(response, "user.name", userName);
        createCookie(response, "user.color", userColor);
        // Обновляем переменные сессии:
        //     - Инкрементируем число посещений.
        //     - Обновляем дату последнего входа.
        updateOrCreateSessionAttribute(session,
                                       "page.welcome.accessedCount",
                x -> x.map(y -> ((Integer)y) + 1).orElse(1));
        updateOrCreateSessionAttribute(session,
                                       "page.welcome.accessedDate",
                x -> Instant.now());

        // Производим редирект на целевую страницу.
        var redirectPage = request.getContextPath() + "/Welcome.jsp";
        response.sendRedirect(response.encodeRedirectURL(redirectPage));
    }

    /**
     * Создает новое куки.
     * @param response Объект ответа.
     * @param name Имя куки.
     * @param value Значение куки.
     */
    private void createCookie(HttpServletResponse response,
                              String name, String value) {
        var c = new Cookie(name, URLEncoder.encode(value,
                                        StandardCharsets.UTF_8));
        c.setMaxAge(60 * 60 * 24);
        response.addCookie(c);
    }

    /**
     * Представляет метод, обновляющий значение переменной HTTP-сессии.
     */
    interface HttpSessionAttributeUpdater {
        Object update(Optional<Object> oldValue);
    }

    /**
     * Обновляет переменную HTTP-сессии.
     * @param session Объект сессии.
     * @param name Имя переменной.
     * @param updater Объект-обновитель.
     */
    private void updateOrCreateSessionAttribute(HttpSession session,
            String name, HttpSessionAttributeUpdater updater) {
        var oldValue = session.getAttribute(name);
        var newValue = updater.update(Optional.ofNullable(oldValue));
        session.setAttribute(name, newValue);
    }

    /**
     * Читает параметр GET-запроса.
     *
     * При отсутствии параметра устанавливает HTTP-код и сообщение.
     * При отсутствии параметра устанавливает HTTP-код и сообщение.
     *
     * @param request Объект запроса.
     * @param response Объект ответа.
     * @param name Имя параметра.
     * @return Значение параметра.
     * @throws IOException Ошибка ввода/вывода.
     */
    private String readParameter(HttpServletRequest request,
                HttpServletResponse response, String name)
                    throws IOException {
        var parameter = request.getParameter(name);
        if(parameter == null) {
            var error = "Parameter \"" + name + "\" is not set";
            response
                .sendError(HttpServletResponse.SC_BAD_REQUEST, error);
        }
        return parameter;
    }
}
\end{minted}

\anonsection{Исходный код Welcome.jsp}

\begin{minted}[fontsize=\footnotesize]{jsp}
<%@ page import="java.util.Optional" %>
<%@ page import="java.util.Arrays" %>
<%@ page import="java.time.Instant" %>
<%@ page contentType="text/html;charset=UTF-8" language="java" %>
<%!
  /**
   * Ищет куки по имени и возвращает его значение.
   * @param request Объект запроса.
   * @param name имя куки.
   * @return Значение куки, если оно найдено.
   */
  Optional<String> findCookieValue(HttpServletRequest request,
        String name) {
    return Arrays.stream(
      request.getCookies())
            .filter(x -> x.getName().equals(name))
            .findFirst()
            .map(Cookie::getValue);
  }
%>
<%
  // Находит значения переменных, сохраненные в куки.
  String userName = findCookieValue(request, "user.name")
                        .orElse("Anon");
  String userColor = findCookieValue(request, "user.color")
                        .orElse("black");
  // Находит значения переменных, сохраненные в сессии.
  Integer accessedCount =
        (Integer)session.getAttribute("page.welcome.accessedCount");
  Instant accessedDate =
        (Instant)session.getAttribute("page.welcome.accessedDate");
%>

<html>
<head>
  <title>Title</title>
  <style>
    em {
      color: <%=userColor%>;
    }
  </style>
</head>
<body>
<h3>
  Hello, <em><%=userName%></em>!
</h3>
<br>
<h3>
  This page has been visited <em><%=accessedCount%></em> time(s).
  Last on <em><%=accessedDate%></em>
</h3>
</body>
</html>
\end{minted}

\end{document}

